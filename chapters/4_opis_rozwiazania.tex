\chapter{OPIS ROZWIĄZANIA}
\label{chapter:opis_rozwiazania}

\section{Architektura systemu}

System składa się z pięciu głównych komponentów: nadajników BLE (ang. \textit{BLE beacon}), aplikacji mobilnej, systemu Wit.ai, serwera oraz bazy danych. Grafikę przedstawiającą architekturę systemu można zobaczyć na rysunku \ref{fig:architecture}.

\begin{figure}[H]
    \centering
    \includesvg[width=\textwidth]{images/architecture.svg}
    \caption{Architektura systemu.}
    \label{fig:architecture}
\end{figure}

Aplikacja mobilna jest odpowiedzialna za odbieranie oraz przetwarzanie sygnału z nadajników. Jej zadaniem jest również interakcja z użytkownikiem i wysyłanie zapytań do serwera. Serwer przetwarza żądania użytkownika, wysyła zapytania do API (ang. \textit{Application Programming Interface}) serwisu Wit.ai, oraz komunikuje się z bazą danych. Baza danych przechowuje dane i modyfikuje lub udostępnia je na żadanie serwera. Komunikacja między aplikacją mobilną a serwerem odbywa się za pomocą protokołu HTTP. Serwer jest odpowiedzialny za przetwarzanie żądań użytkownika, a także za komunikację z bazą danych. Baza danych przechowuje dane o produktach, użytkownikach, koszykach, sklepach itp.

\section{Baza danych}

\subsection{Opis bazy danych}

Baza danych została zaimplementowana w PostgreSQL. Wybór tej bazy danych wynika z jej wszechstronności, wydajności oraz możliwości łatwego skalowania. Baza danych przechowuje informacje o produktach, użytkownikach, koszykach oraz sklepach. Schemat bazy danych przedstawia rysunek \ref{fig:database}.

Struktura bazy danych została zaprojektowana w sposób modularny, umożliwiając efektywne zarządzanie danymi dotyczącymi sklepów, użytkowników oraz produktów. Główną tabelą bazy danych jest tabela \textit{stores}, która przechowuje informacje o sklepach, takie jak nazwa, współrzędne geograficzne oraz miasto. Związek tej tabeli z tabelą \textit{sections} umożliwia podział sklepów na sekcje, które z kolei są przypisane do tabeli \textit{categories}, zawierającej dane o kategoriach produktów.

Produkty są przechowywane w tabeli \textit{products}, gdzie każdy rekord zawiera szczegóły takie jak nazwa, opis, cena, dostępność, ilość oraz jednostka miary, przechowywana w tabeli \textit{units}. Relacje między tabelami \textit{categories} i \textit{p}roducts pozwalają na przypisanie każdego produktu do konkretnej kategorii, co ułatwia organizację i wyszukiwanie danych.

Użytkownicy systemu są reprezentowani w tabeli \textit{users}, gdzie zapisywane są ich dane personalne, takie jak imię, nazwisko, adres e-mail oraz zaszyfrowane hasło. Każdy użytkownik może posiadać wiele koszyków zakupowych, co jest odzwierciedlone w tabeli \textit{carts}, przechowującej informacje o koszykach, takie jak data utworzenia i powiązanie z użytkownikiem. 
Szczegóły dotyczące zawartości koszyków są zapisane w tabeli \textit{cart\_items}, która łączy produkty z koszykami i zawiera informacje o liczbie sztuk danego produktu.

Relacje pomiędzy tabelami są realizowane za pomocą kluczy obcych, z zastosowaniem reguły ON DELETE CASCADE, co zapewnia integralność danych oraz automatyczne usuwanie powiązanych rekordów w przypadku usunięcia danych z tabel nadrzędnych. Taka organizacja umożliwia łatwe skalowanie bazy danych oraz wspiera utrzymanie spójności danych w systemie.

\begin{figure}[H]
    \includesvg[width=\textwidth]{images/database.svg}
    \caption{Schemat bazy danych.}
    \label{fig:database}
\end{figure}

\subsection{Szczegółowy opis tabel}

\subsubsection{Tabela stores}
\begin{itemize}
\item store\_id - SERIAL PRIMARY KEY: Unikalny identyfikator każdego sklepu.
\item store\_name - VARCHAR(255) NOT NULL: Nazwa sklepu.
\item latitude - VARCHAR(255) NOT NULL: Szerokość geograficzna określająca położenie sklepu.
\item longitude - VARCHAR(255) NOT NULL: Długość geograficzna określająca położenie sklepu.
\item city - VARCHAR(255) NOT NULL: Miasto, w którym znajduje się sklep.
\end{itemize}

\subsubsection{Tabela sections}
\begin{itemize}
\item section\_id - SERIAL PRIMARY KEY: Unikalny identyfikator sekcji sklepu.
\item section\_name - VARCHAR(255) NOT NULL: Nazwa sekcji w sklepie.
\item store\_id - INT REFERENCES stores(store\_id) ON DELETE CASCADE: Klucz obcy wskazujący sklep, do którego należy sekcja.
\end{itemize}

\subsubsection{Tabela categories}
\begin{itemize}
\item category\_id - SERIAL PRIMARY KEY: Unikalny identyfikator kategorii.
\item category\_name - VARCHAR(255) NOT NULL: Nazwa kategorii produktów.
\item section\_id - INT REFERENCES sections(section\_id) ON DELETE CASCADE: Klucz obcy wskazujący sekcję, do której przypisana jest kategoria.
\end{itemize}

\subsubsection{Tabela units}
\begin{itemize}
\item unit\_id - SERIAL PRIMARY KEY: Unikalny identyfikator jednostki miary.
\item unit\_name - VARCHAR(50) NOT NULL: Pełna nazwa jednostki miary (np. “kilogram”).
\item unit\_symbol - VARCHAR(10) NOT NULL: Skrót jednostki miary (np. “kg”).
\end{itemize}

\subsubsection{Tabela products}
\begin{itemize}
\item product\_id - SERIAL PRIMARY KEY: Unikalny identyfikator produktu.
\item name - VARCHAR(255) NOT NULL: Nazwa produktu.
\item description - TEXT: Opis produktu.
\item price - DECIMAL(10,2) NOT NULL: Cena produktu w formacie dziesiętnym (np. 123.45).
\item category\_id - INT REFERENCES categories(category\_id) ON DELETE CASCADE: Klucz obcy wskazujący kategorię, do której należy produkt.
\item availability - VARCHAR(50) NOT NULL: Status dostępności produktu (np. “w magazynie”).
\item amount - DECIMAL(10,2) NOT NULL: Ilość dostępna w magazynie.
\item unit\_id - INT REFERENCES units(unit\_id) ON DELETE CASCADE: Klucz obcy wskazujący jednostkę miary produktu.
\end{itemize}

\subsubsection{Tabela users}
\begin{itemize}
\item user\_id - SERIAL PRIMARY KEY: Unikalny identyfikator użytkownika.
\item email - VARCHAR(255) UNIQUE NOT NULL: Adres e-mail użytkownika.
\item password - VARCHAR(255) NOT NULL: Hasło użytkownika (w formie zaszyfrowanej).
\item first\_name - VARCHAR(50) NOT NULL: Imię użytkownika.
\item last\_name - VARCHAR(50) NOT NULL: Nazwisko użytkownika.
\end{itemize}

\subsubsection{Tabela carts}
\begin{itemize}
\item cart\_id - SERIAL PRIMARY KEY: Unikalny identyfikator koszyka.
\item user\_id - INT REFERENCES users(user\_id) ON DELETE CASCADE: Klucz obcy wskazujący użytkownika, do którego należy koszyk.
\item creation\_date - TIMESTAMP DEFAULT CURRENT\_TIMESTAMP: Data i czas utworzenia koszyka.
\end{itemize}

\subsubsection{Tabela cart\_items}
\begin{itemize}
\item cart\_item\_id - SERIAL PRIMARY KEY: Unikalny identyfikator pozycji w koszyku.
\item cart\_id - INT REFERENCES carts(cart\_id) ON DELETE CASCADE: Klucz obcy wskazujący koszyk, do którego należy pozycja.
\item product\_id - INT REFERENCES products(product\_id) ON DELETE CASCADE: Klucz obcy wskazujący produkt dodany do koszyka.
\item quantity - INT NOT NULL: Liczba sztuk danego produktu w koszyku.
\end{itemize}

tekst