\chapter{OPIS TECHNICZNY}
\label{chapter:opis_techniczny}


\section{Aplikacja wit.ai}
W ramach projektu została stworzona aplikacja w systemie \textit{wit.ai}. Wit.ai to platforma do tworzenia interfejsów interaktywnych, która pozwala na budowanie aplikacji, które rozumieją naturalny język. Wit.ai pozwala na tworzenie modeli językowych, które są w stanie rozpoznawać intencje użytkownika na podstawie zdefiniowanych przez programistę fraz. Aplikacja ta jest wykorzystywana w projekcie do rozpoznawania intencji użytkownika na podstawie zdefiniowanych przez programistę fraz.

\subsection{Intencje i encje}

W celu nauczenia modelu językowego aplikacji wit.ai, należy zdefiniować intencje (ang. \textit{intents}), które mają być rozpoznawane przez aplikację. Intencje to frazy, które użytkownik może napisać, a które mają być zrozumiane przez aplikację. Każda intencja może zawierać wiele przykładów fraz, które są z nią związane. Przykładowe intencje, które zostały zdefiniowane w aplikacji wit.ai to:
\begin{itemize}
    \item \textit{add\_product\_to\_cart} - intencja dodania produktu do koszyka,
    \item \textit{remove\_product\_from\_cart} - intencja usunięcia produktu z koszyka,
    \item \textit{check\_cart} - intencja sprawdzenia zawartości koszyka,
    \item \textit{check\_item\_prices} - intencja sprawdzenia cen produktów,
    \item \textit{check\_item\_price\_in\_store} - intencja sprawdzenia ceny produktu w sklepie
\end{itemize}

Poza intencjami, w aplikacji wit.ai definiuje się również encje (ang. \textit{entities}). Encje to frazy, które mają być rozpoznawane przez aplikację jako konkretne wartości. Encje pomagają również w wykryciu intencji użytkownika. Przykładowe encje, które zostały zdefiniowane w aplikacji wit.ai to:
\begin{itemize}
    \item \textit{product} - encja reprezentująca nazwę produktu,
    \item \textit{store} - encja reprezentująca nazwę sklepu,
    \item \textit{view} - encja reprezentująca widok w aplikacji,
    \item \textit{category} - encja reprezentująca nazwę kategorii produktów
\end{itemize}

Po zdefiniowaniu intencji i encji, aplikacja wit.ai pozwala na trenowanie modelu językowego. Trenowanie modelu polega na przesłaniu do aplikacji wit.ai przykładów fraz, które mają być rozpoznawane przez aplikację. Po przesłaniu przykładów, aplikacja wit.ai trenuje model językowy. Po wstępnym treningu, aplikacja stara się sama sugerować intencje w procesie trenowania modelu. Pozwala to na sprawdzanie w czasie rzeczywistym, czy model językowy poprawnie rozpoznaje encje i intencje. Przykładowe frazy wykorzystane w procesie trenowania modelu to:
\begin{itemize}
    \item \textit{Where to buy apples}
    \item \textit{Remove cheese from cart},
    \item \textit{My app is stuck on loading},
    \item \textit{Is there dairy in castorama?}
\end{itemize}

\subsection{Kreator}

Wytrenowany model należy zaprogramować. W tym celu wit.ai udostępnia kreator (ang. \textit{composer}), który pozwala na zdefiniowanie akcji, które mają być wykonywane po rozpoznaniu intencji przez aplikację. 
