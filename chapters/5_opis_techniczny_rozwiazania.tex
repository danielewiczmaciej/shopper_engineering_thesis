\chapter{OPIS TECHNICZNY}

\section{Architektura systemu}

System składa się z pięciu głównych komponentów: nadajników BLE (ang. \textit{BLE beacon}), aplikacji mobilnej, systemu Wit.ai, serwera oraz bazy danych. Grafikę przedstawiającą architekturę systemu można zobaczyć na rysunku \ref{fig:architecture}.

\begin{figure}[ht]
    \centering
    \includesvg[width=\textwidth]{images/architecture.svg}
    \caption{Architektura systemu.}
    \label{fig:architecture}
\end{figure}

Aplikacja mobilna jest odpowiedzialna za odbieranie oraz przetwarzanie sygnału z nadajników. Jej zadaniem jest również interakcja z użytkownikiem i wysyłanie zapytań do serwera. Serwer przetwarza żądania użytkownika, wysyła zapytania do API (ang. \textit{Application Programming Interface}) serwisu Wit.ai, oraz komunikuje się z bazą danych. Baza danych przechowuje dane i modyfikuje lub udostępnia je na żadanie serwera. Komunikacja między aplikacją mobilną a serwerem odbywa się za pomocą protokołu HTTP. Serwer jest odpowiedzialny za przetwarzanie żądań użytkownika, a także za komunikację z bazą danych. Baza danych przechowuje dane o produktach, użytkownikach, koszykach, sklepach itp.

