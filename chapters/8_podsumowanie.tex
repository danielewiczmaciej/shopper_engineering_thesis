\chapter{PODSUMOWANIE}
\label{chapter:podsumowanie}

\section{Przyszły rozwój aplikacji}

Po zakończeniu pracy nad projektem w ramach pracy inżynierskiej, aplikacja jest gotowa do wdrożenia. W ramach przyszłego rozwoju aplikacji można by było zaimplementować kilka dodatkowych funkcjonalności, które rozszerzyłyby możliwości aplikacji. Poniżej przedstawiono kilka z nich.

\subsection{Panel Administratora}
W ramach przyszłego rozwoju aplikacji, warto byłoby dodać panel administratora, który pozwoliłby na zarządzanie użytkownikami, ich uprawnieniami, a także na zarządzanie treścią aplikacji. W panelu administratora można by było dodawać nowe kategorie, podkategorie, a także zarządzać treściami w ramach tych kategorii.
Należy rozróżnić panel administratora dostępny dla managera sklepu, od panelu dostępnego dla administratora systemu. W ramach panelu managera mogłyby zostać zaimplementowane następujące funkcje:
\begin{itemize}
    \item Zarządzanie asortymentem sklepu
    \item Zarządzanie użytkownikami w zakresie sklepu
    \item Możliwość zgłośenia problemu z aplikacją
\end{itemize}

Panel dostępny dla administratora systemu rozszerzałby powyższy panel o dodatkowe funkcje, takie jak:
\begin{itemize}
    \item Zarządzanie użytkownikami w zakresie całego systemu
    \item Zarządzanie kategoriami i podkategoriami
    \item Zarządzanie treściami w ramach kategorii
    \item Zarządzanie zgłoszeniami problemów z aplikacją
    \item Zarządzanie panelami dostępowymi
\end{itemize}

\subsection{Rozwinięcie modelu językowego}
W ramach przyszłego rozwoju aplikacji, warto byłoby rozwinąć model językowy, który pozwoliłby na bardziej zaawansowane rozpoznawanie intencji użytkownika. W ramach rozwoju modelu językowego można by było zaimplementować dodatkowe intencje, które pozwoliłyby na bardziej zaawansowane zapytania użytkownika. Przykładowe intencje, które można by było zaimplementować to:
\begin{itemize}
    \item Zapytanie o skład produktu
    \item Zapytanie o listę składników do przygotowania dania
    \item Zapytanie o promocje w sklepie
    \item Zapytanie o godziny otwarcia sklepu
\end{itemize}


\subsection{Rozwój aplikacji mobilnej}

\subsubsection{Inwentaż}

Kolejnym planowanym rozszerzeniem aplikacji mobilnej jest funkcjonalność inwentarza. Miałoby to na celu ułatwienie układania listy zakupów. Użytkownik dodawałby do inwentarza produkty, które posiada już w domu. Następnie ustaliłby ilośc każdego z produktów, jaką chciałby mieć na codzień w domu.
Na tej podstawie, aplikacja generowałaby listę produktów, których brakuje w domu. 

\subsubsection{Planer zakupów i posiłków}

Następnym możliwym rozszerzeniem jest planer zakupów. W ramach planera, użytkownik mógłby ustalić posiłki, które chciałby zjeść w tym tygodniu. W tym celu aplikacja musiałaby być rozszerzona o funkcję sprawdzania składników poszczególnych posiłków. Na tej podstawie, aplikacja generowałaby listę zakupów, która zawierałaby produkty, które są potrzebne do przygotowania posiłków na cały tydzień.