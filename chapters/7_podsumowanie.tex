\chapter{PODSUMOWANIE}
\label{chapter:podsumowanie}

\section{Napotkane wyzwania}

Podczas pracy nad projektem napotkano kilka wyzwań, które utrudniły proces implementacji. Poniżej przedstawiono najważniejsze z nich.

\subsection{Obsługa nadajników BLE}
Dla każdego członka projektu, technologia Bluetooth Low Energy była nowością. W związku z tym, konieczne było zapoznanie się z dokumentacją techniczną oraz zasadami działania tej technologii. W trakcie implementacji, napotkano kilka problemów związanych z obsługą nadajników BLE, które wymagały głębszej analizy i zrozumienia zasad działania tej technologii.
Pierwszą przeszkodą było zasilenie nadajnika. Jako, że urządzenie ma jedynie piny VCC, GND, TX i RX, konieczne było zasilanie nadajnika z zewnętrznego źródła. W tym celu wypożyczono sprzęt w postaci płytek stykowych, przewodów połączeniowych oraz urządzenia Raspberry Pi. Następnie podłączono nadajnik do Raspberry Pi. 
Zasilone urządzenie należało następnie wyszukać wśród dostępnych urządzeń bluetooth. Do zidentyfikowania sygnału i obsługi urządzenia, użyto dokumentu technicznego opisującego interfejs nadajnika. \cite{hm10datasheet}

\subsection{Praca w zespole}
Kolejnym z wyzwań, które napotkano już na początku projektu była organizacja pracy zespołowej. Każdy członek zespołu miał swoje zadanie do wykonania, lecz wytwarzane komponenty były zależne od kodu innych członków zespołu. W związku z tym, konieczne było zapewnienie ciągłej komunikacji w zespole.
W tym celu ustanowiono cotygodniowe spotkanie mające na celu omówienie postępów w pracy, a także rozwiązanie napotkanych problemów. W trakcie spotkań omawiano również plany na kolejny tydzień, a także ustalano priorytety zadań. Takie podejście pozwoliło na efektywną pracę zespołową i na osiągnięcie celu projektu.

\section{Przyszły rozwój aplikacji}

Po zakończeniu pracy nad projektem w ramach pracy inżynierskiej, aplikacja jest gotowa do wdrożenia. W ramach przyszłego rozwoju aplikacji można by było zaimplementować kilka dodatkowych funkcjonalności, które rozszerzyłyby możliwości aplikacji. Poniżej przedstawiono kilka z nich.

\subsection{Panel Administratora}
W ramach przyszłego rozwoju aplikacji, warto byłoby dodać panel administratora, który pozwoliłby na zarządzanie użytkownikami, ich uprawnieniami, a także na zarządzanie treścią aplikacji. W panelu administratora można by było dodawać nowe kategorie, podkategorie, a także zarządzać treściami w ramach tych kategorii.
Należy rozróżnić panel administratora dostępny dla managera sklepu, od panelu dostępnego dla administratora systemu. W ramach panelu managera mogłyby zostać zaimplementowane następujące funkcje:
\begin{itemize}
    \item Zarządzanie asortymentem sklepu
    \item Zarządzanie użytkownikami w zakresie sklepu
    \item Możliwość zgłośenia problemu z aplikacją
\end{itemize}
Panel dostępny dla administratora systemu rozszerzałby powyższy panel o dodatkowe funkcje, takie jak:
\begin{itemize}
    \item Zarządzanie użytkownikami w zakresie całego systemu
    \item Zarządzanie kategoriami i podkategoriami
    \item Zarządzanie treściami w ramach kategorii
    \item Zarządzanie zgłoszeniami problemów z aplikacją
    \item Zarządzanie panelami dostępowymi
\end{itemize}

\subsection{Rozwinięcie modelu językowego}
W ramach przyszłego rozwoju aplikacji, warto byłoby rozwinąć model językowy, który pozwoliłby na bardziej zaawansowane rozpoznawanie intencji użytkownika. W ramach rozwoju modelu językowego można by było zaimplementować dodatkowe intencje, które pozwoliłyby na bardziej zaawansowane zapytania użytkownika. Przykładowe intencje, które można by było zaimplementować to:
\begin{itemize}
    \item Zapytanie o skład produktu
    \item Zapytanie o listę składników do przygotowania dania
    \item Zapytanie o promocje w sklepie
    \item Zapytanie o godziny otwarcia sklepu
\end{itemize}


\subsection{Rozwój aplikacji mobilnej}

\subsubsection{inwentarz}

Kolejnym planowanym rozszerzeniem aplikacji mobilnej jest funkcjonalność inwentarza. Miałoby to na celu ułatwienie układania listy zakupów. Użytkownik dodawałby do inwentarza produkty, które posiada już w domu. Następnie ustaliłby ilośc każdego z produktów, jaką chciałby mieć na codzień w domu.
Na tej podstawie, aplikacja generowałaby listę produktów, których brakuje w domu. 

\subsubsection{Planer zakupów i posiłków}

Następnym możliwym rozszerzeniem jest planer zakupów. W ramach planera, użytkownik mógłby ustalić posiłki, które chciałby zjeść w tym tygodniu. W tym celu aplikacja musiałaby być rozszerzona o funkcję sprawdzania składników poszczególnych posiłków. Na tej podstawie, aplikacja generowałaby listę zakupów, która zawierałaby produkty, które są potrzebne do przygotowania posiłków na cały tydzień.

\section{Wnioski}

\subsection{Osiągnięte cele}

W ramach pracy inżynierskiej udało się zrealizować wszystkie cele postawione przed projektem. Aplikacja spełnia wszystkie wymagania funkcjonalne, które zostały zdefiniowane na początku projektu. W związku z tym, można uznać, że projekt zakończył się sukcesem.

\subsection{Wartość praktyczna}

Aplikacja ma duże znaczenie praktyczne zarówno dla użytkowników indywidualnych, jak i dla właścicieli sklepów. Umożliwiając łatwiejsze i bardziej efektywne zakupy, aplikacja przyczynia się do zwiększenia komfortu klientów, szczególnie tych, którzy borykają się z ograniczeniami ruchowymi, wzrokowymi lub innymi problemami zdrowotnymi. Funkcja nawigacji po sklepie z wykorzystaniem algorytmów optymalizacyjnych pozwala zaoszczędzić czas i zmniejszyć stres związany z wyszukiwaniem produktów w dużych placówkach. Dodatkowo, dzięki funkcji generowania kodów QR, proces finalizacji zakupów został znacznie uproszczony, co jest korzystne zarówno dla klientów, jak i dla sklepów, które mogą ograniczyć kolejki i usprawnić działanie kas samoobsługowych.

Dla branży handlowej aplikacja stanowi innowacyjne rozwiązanie, które może przyciągnąć nowych klientów, poprawić wizerunek firmy i zwiększyć jej konkurencyjność. Rozwiązanie to jest szczególnie cenne w kontekście współczesnych trendów, takich jak personalizacja usług czy dążenie do inkluzywności. Możliwość integracji aplikacji z istniejącymi systemami sklepowymi, jak również jej potencjał do działania w wielu sklepach, a nie tylko w jednej sieci, zwiększa elastyczność wdrożenia.

Oprócz zastosowań komercyjnych, aplikacja może znaleźć zastosowanie w programach społecznych wspierających osoby z niepełnosprawnościami, zwiększając ich samodzielność i komfort życia. Dzięki temu projekt wpisuje się w ideę społecznej odpowiedzialności biznesu (CSR) i może stanowić wzór dla innych rozwiązań technologicznych dążących do integracji i wsparcia osób wykluczonych.