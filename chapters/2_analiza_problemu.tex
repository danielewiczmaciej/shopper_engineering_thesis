\chapter{ANALIZA PROBLEMU}

\section{Nawigacja wewnątrz budynku}

\section{Przetwarzanie języka naturalnego}

„Natural Language Processing (NLP) to dziedzina badań i zastosowań, która eksploruje, jak komputery mogą rozumieć i manipulować naturalnym językiem w formie tekstu lub mowy w celu wykonania użytecznych zadań” \\ \cite{Chowdhary2020} 


NLP znajduje zastosowanie w różnych obszarach, takich jak tłumaczenie maszynowe, przetwarzanie tekstu, streszczanie, interfejsy użytkownika, rozpoznawanie mowy i systemy ekspertowe. W szczególności w aplikacjach handlowych NLP może poprawić wyszukiwanie informacji i interakcje z użytkownikami.

Budowanie systemów NLP obejmuje analizę na kilku poziomach:
\begin{enumerate}
    \item Foniczny i fonologiczny: wymowa i dźwięk.
    \item Morfologiczny: analiza najmniejszych jednostek językowych.
    \item Syntaktyczny: struktura zdań.
    \item Semantyczny: znaczenie słów i zdań.
    \item Dyskursywny i pragmatyczny: kontekst i wiedza zewnętrzna (Liddy, 1998; Feldman, 1999). \cite{Chowdhary2020}
\end{enumerate}

Natural Language Interfaces (NLI) umożliwiają użytkownikom zadawanie pytań w języku naturalnym, co może być szczególnie przydatne w aplikacjach zakupowych, np. „Gdzie znajdę makaron?” lub „Dodaj do koszyka mleko”.
