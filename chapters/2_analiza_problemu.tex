\chapter{ANALIZA PROBLEMU}
\label{chapter:analiza_problemu}

\section{Trasowanie po sklepie}

Sklep można przedstawić jako graf, w którym wierzchołki reprezentują sekcje sklepu, a te sąsiednie połączone są w grafie krwędzią. Sekcje to krótkie fragmenty alejki o długości 1–2 metrów, do których przypisane są konkretne produkty w bazie danych. Dzięki temu podejściu cały sklep można traktować jako graf, a to z kolei pozawala na zastosowanie dobrze znanych algorytmów do znajdowania ścieżek.

Przciętny graf sklepu ma prostą i powtarzalną strukturę: składa się z kilku długich, równoległych alejek, połączonych jedną poziomą alejką na dole i drugą u góry. Wagi krawędzi, są tak na prawdę odzwierciedlonymi wartościami odległości od obu środków sąsiednich sekcji. Są więc one często stałe, a wyjątki pojawiają się głównie na przejściach między alejkami i zakrętach.

Proces trasowania otrzymuje na wejściu listę wierzchołków zmapowaną z listy zakupów użytkownika. System nawigacji musi znaleźć najkrótszą drogę w grafie, zaczynając od aktualnej pozycji użytkownika (dostarczanej przez system pozycjonowania) i przechodząc przez wszystkie wierzchołki z listy.

\subsection{Podobieństwa i różnice z problemem komiwojażera (TSP)}

Na pierwszy rzut oka nasz mechanizm może kojarzyć się problem komiwojażera (\textit{Traveling Salesman Problem, TSP}), w którym zadaniem jest odwiedzenie każdego punktu na grafie dokładnie raz i powrót do punktu startowego. Jak wskazano w pracy Gutin i Punnen (2002): „Problem komiwojażera jest jednym z najbardziej znanych problemów optymalizacji kombinatorycznej, a jego rozwiązanie wymaga innowacyjnych podejść ze względu na jego złożoność” \cite{Gutin2002}. Na szczęście, te drobne różnice sprawiają, że problem trasowania po sklepie jest znacznie prostszy:
\begin{itemize}
    \item W trasowaniu po sklepie krawędzie i wierzchołki mogą być odwiedzane wielokrotnie, jeśli wymaga tego optymalna trasa.
    \item Nie ma konieczności powrotu do punktu początkowego, co znacząco upraszcza problem.
    \item Graf sklepu ma bardzo regularną strukturę i często stałe wagi krawędzi, co pozwala na duże optymalizacje.
\end{itemize}

\subsection{Wnioski}

Specyfika grafu sklepu oraz brak potrzeby dokładnego odwzorowania problemu TSP umożliwia wykorzystanie prostszych algorytmów, takich jak algorytm Dijkstry z powodzeniem. Dodatkowo przy dłuższych listach z produktami można zastosować heurystyki, takie jak algorytm 2-opt, które dodatkowo potrafi poporawić trasę. Jak podkreślił Croes (1958): „Algorytm 2-opt oferuje prostą, ale skuteczną metodę poprawy tras dla problemów trasowania w grafach, co czyni go odpowiednim w praktycznych zastosowaniach” \cite{Croes1958}.


\section{Przetwarzanie języka naturalnego}

„Przetwarzanie języka naturalnego (ang. \textit{Natural Language Processing}) to dziedzina badań i zastosowań, która eksploruje, jak komputery mogą rozumieć i manipulować naturalnym językiem w formie tekstu lub mowy w celu wykonania użytecznych zadań” \\ \cite{Chowdhary2020} 


NLP znajduje zastosowanie w różnych obszarach, takich jak tłumaczenie maszynowe, przetwarzanie tekstu, streszczanie, interfejsy użytkownika, rozpoznawanie mowy i systemy ekspertowe. W szczególności w aplikacjach handlowych NLP może poprawić wyszukiwanie informacji i interakcje z użytkownikami.

Budowanie systemów NLP obejmuje analizę na kilku poziomach:
\begin{enumerate}
    \item Foniczny i fonologiczny: wymowa i dźwięk.
    \item Morfologiczny: analiza najmniejszych jednostek językowych.
    \item Syntaktyczny: struktura zdań.
    \item Semantyczny: znaczenie słów i zdań.
    \item Dyskursywny i pragmatyczny: kontekst i wiedza zewnętrzna (Liddy, 1998; Feldman, 1999). \cite{Chowdhary2020}
\end{enumerate}

Natural Language Interfaces (NLI) umożliwiają użytkownikom zadawanie pytań w języku naturalnym, co może być szczególnie przydatne w aplikacjach zakupowych, np. „Gdzie znajdę makaron?” lub „Dodaj do koszyka mleko”. W przypadku aplikacji będącej tematem pracy, NLI zostało wykorzystane również do pomocy w obsłudze aplikacji.
