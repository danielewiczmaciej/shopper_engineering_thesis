\chapter{ANALIZA PROBLEMU}
\label{chapter:analiza_problemu}

\section{Trasowanie po sklepie}

\subsection{Wprowadzenie do problemu trasowania}

Trasowanie po sklepie to zadanie polegające na znalezieniu optymalnej ścieżki, która pozwoli użytkownikowi odwiedzić wszystkie poszukiwane produkty w sposób możliwie efektywny. Sklep można przedstawić jako graf, w którym wierzchołki reprezentują sekcje sklepu, a te sąsiednie połączone są w grafie krawędzią. Sekcje to krótkie fragmenty alejki o długości 1–2 metrów, do których przypisane są konkretne produkty w bazie danych. Dzięki temu podejściu cały sklep można traktować jako graf, a to z kolei pozawala na zastosowanie dobrze znanych algorytmów do znajdowania ścieżek.

Proces trasowania otrzymuje na wejściu listę wierzchołków zmapowaną z listy zakupów użytkownika. System nawigacji musi znaleźć najkrótszą drogę w grafie, zaczynając od aktualnej pozycji użytkownika (dostarczanej przez system pozycjonowania) i przechodząc przez wszystkie wierzchołki z listy.

\subsection{Modelowanie sklepu jako graf}
Przeciętny graf sklepu ma prostą i powtarzalną strukturę: składa się z kilku długich, równoległych alejek, połączonych jedną poziomą alejką na dole i drugą u góry. Wagi krawędzi odzwierciedlają odległości pomiędzy sekcjami – często są one stałe, a wyjątki pojawiają się głównie na przejściach między alejkami i zakrętach. Dzięki temu sklep można przedstawić jako graf o regularnej strukturze, co ułatwia zastosowanie klasycznych algorytmów grafowych.

\subsection{Podobieństwa i różnice z problemem komiwojażera (TSP)}
Na pierwszy rzut oka trasowanie po sklepie może kojarzyć się z problemem komiwojażera (\textit{Traveling Salesman Problem, TSP}), w którym zadaniem jest odwiedzenie każdego punktu na grafie dokładnie raz i powrót do punktu startowego. Jak wskazano w pracy Gutin i Punnen (2002): „Problem komiwojażera jest jednym z najbardziej znanych problemów optymalizacji kombinatorycznej, a jego rozwiązanie wymaga innowacyjnych podejść ze względu na jego złożoność” \cite{Gutin2002}.

Jednak istnieją istotne różnice, które sprawiają, że problem trasowania po sklepie jest prostszy:
\begin{itemize}
    \item Krawędzie i wierzchołki w sklepie mogą być odwiedzane wielokrotnie, jeśli wymaga tego optymalna trasa.
    \item Nie ma konieczności powrotu do punktu początkowego, co znacząco upraszcza problem.
    \item Graf sklepu ma bardzo regularną strukturę i często stałe wagi krawędzi, co pozwala na duże optymalizacje.
\end{itemize}

Badania Gu, Lo i Niemegeers (2009) wskazują, że struktury wewnętrzne (np. układ alejek w sklepach) różnią się od złożonych grafów znanych z TSP, przez co podejście do optymalizacji trasy może być znacznie uproszczone \cite{Gu2009}.

\subsection{Algorytmy i heurystyki trasowania}

\subsubsection{Algorytm Dijkstry i najkrótsze ścieżki}
Najprostszym podejściem do problemu trasowania po sklepie jest potraktowanie go jako problem znajdowania najkrótszej ścieżki na grafie. Klasycznym algorytmem rozwiązującym ten problem jest algorytm Dijkstry. Jak podają Cormen i in. (2009): „Algorytm Dijkstry pozwala na efektywne znalezienie najkrótszej ścieżki w grafie z nieujemnymi wagami krawędzi” \cite{Cormen2009}.

Wykorzystując algorytm Dijkstry, można obliczyć najkrótszą trasę pomiędzy kolejnymi produktami z listy zakupów. W celu obsłużenia wielu punktów docelowych stosuje się strategię sekwencyjnego wywoływania algorytmu dla kolejnych punktów lub łączenia technik heurystycznych.

\subsubsection{Heurystyki optymalizacyjne: algorytm 2-opt i inne podejścia}
Przy dłuższych listach produktów i większych sklepach znalezienie globalnie optymalnej trasy może być trudne obliczeniowo. W takim przypadku stosuje się heurystyki, które pozwalają na poprawę pierwotnie znalezionej ścieżki. Jedną z popularnych metod jest algorytm 2-opt, który poprzez iteracyjne zamiany krawędzi potrafi wyeliminować nielogiczne „skrzyżowania” ścieżek, skracając jej długość.

Jak podkreślił Croes (1958): „Algorytm 2-opt oferuje prostą, ale skuteczną metodę poprawy tras dla problemów trasowania w grafach, co czyni go odpowiednim w praktycznych zastosowaniach” \cite{Croes1958}.

Inne heurystyki, takie jak algorytmy zachłanne, algorytmy genetyczne czy metaheurystyki typu tabu search, mogą również okazać się przydatne w skomplikowanych scenariuszach. Lawler i in. (1985) zwracają uwagę, że „różnorodność podejść heurystycznych pozwala na elastyczne dostosowanie do specyfiki danego problemu trasowania, w tym ograniczeń środowiskowych i preferencji użytkowników” \cite{Lawler1985}.

\subsection{Czynniki środowiskowe i praktyczne ograniczenia}
W praktyce trasowanie po sklepie nie jest zadaniem wykonywanym w idealnych warunkach. Istnieje wiele czynników utrudniających optymalizację trasy, takich jak:
\begin{itemize}
    \item Zmienne warunki w sklepie: zmieniające się położenie produktów, remonty i blokady.
    \item Interakcje z innymi klientami i ruchem w alejkach.
    \item Ograniczona dokładność systemów pozycjonowania wewnątrz budynków, które dostarczają informacje o bieżącej pozycji użytkownika.
\end{itemize}

Larson, Bradlow i Fader (2005) zauważają, że „rzeczywiste ścieżki klientów często odbiegają od optymalnych tras wyliczonych przez algorytm, co wynika z czynników psychologicznych, zmiennych preferencji oraz uwarunkowań przestrzennych” \cite{Larson2005}.

\subsection{Dostrajanie i walidacja rozwiązań}
Aby zapewnić jak najlepsze dopasowanie trasowania do warunków panujących w sklepie, konieczne jest dostrajanie parametrów algorytmów i heurystyk. W praktyce oznacza to przeprowadzanie testów, zbieranie danych o realnych ścieżkach klientów oraz uwzględnianie aktualnej konfiguracji sklepu. Badania nad optymalizacją tras po sklepach często opierają się na analizie danych zebranych na podstawie zachowań konsumentów, co pozwala na kalibrację modeli.

Jak zauważają Larson, Bradlow i Fader (2005): „Analiza ścieżek klientów w supermarketach ujawnia pewną regularność ruchu oraz powtarzalność sekwencji odwiedzanych produktów, co można wykorzystać do efektywnego modelowania i optymalizacji tras” \cite{Larson2005}.

\subsection{Wnioski}
Rozszerzona analiza trasowania po sklepie pokazuje, że problem ten można efektywnie rozwiązywać dzięki modelowaniu środowiska jako grafu oraz stosowaniu znanych metod optymalizacyjnych, takich jak algorytm Dijkstry czy heurystyki typu 2-opt. Choć problem na pierwszy rzut oka przypomina TSP, mniej rygorystyczne założenia oraz regularna struktura sklepu znacząco go upraszczają. Mimo to, praktyczne czynniki, takie jak zmieniające się warunki środowiskowe i ograniczona precyzja systemów IPS, stanowią wyzwania, które wymagają elastycznych i adaptacyjnych podejść.

\section{System pozycjonowania wewnątrz budynków}

\subsection{Wprowadzenie do problematyki pozycjonowania wewnętrznego}
Pozycjonowanie wewnętrzne, znane jako Indoor Positioning System (IPS), jest technologią umożliwiającą określenie lokalizacji użytkownika w przestrzeniach zamkniętych, gdzie sygnał GPS jest niedostępny lub niewystarczający. Rozwiązania tego typu zyskują na popularności w wielu branżach, takich jak handel detaliczny, logistyka, edukacja czy zarządzanie budynkami. W szczególności w sklepach detalicznych IPS mogą dać wsparcie klientom, wskazując drogę do poszukiwanych produktów, co może przyczynić się do poprawy komfortu zakupów oraz zwiększenia sprzedaży.

Jak zauważają Zafari, Gkelias i Leung (2019): „Choć osiągnięto znaczący postęp, pozycjonowanie wewnętrzne wciąż pozostaje nierozwiązanym problemem ze względu na złożoność środowisk wewnętrznych” \cite{Zafari2019}. W podobnym tonie wypowiada się Mautz (2012): „Środowiska wewnętrzne charakteryzują się propagacją wielodrogową, tłumieniem, warunkami bez linii widzenia oraz interferencjami, z których wszystkie pogarszają wydajność systemów pozycjonowania” \cite{Mautz2012}.

Rozwój IPS wiąże się jednak z licznymi wyzwaniami technicznymi i środowiskowymi. Kluczowymi problemami są brak precyzji wynikający z niestabilności sygnałów radiowych, złożoność infrastruktury wymaganej do funkcjonowania systemu oraz wysokie koszty implementacji i utrzymania.

\subsection{Technologia Bluetooth Low Energy i jej zastosowanie w IPS}
Bluetooth Low Energy (BLE) jest jedną z najczęściej wykorzystywanych technologii w IPS dzięki niskiemu zużyciu energii, relatywnie niskiemu kosztowi wdrożenia oraz szerokiej kompatybilności z urządzeniami mobilnymi. Nadajniki BLE, znane jako beacony, emitują sygnały o określonej mocy, które mogą być odbierane przez urządzenia użytkownika. Na podstawie wskaźnika siły sygnału (RSSI – Received Signal Strength Indicator) możliwe jest oszacowanie odległości pomiędzy urządzeniem a nadajnikiem.

Jak podkreślają Faragher i Harle (2015): „BLE to technologia niskokosztowa i niskoenergetyczna, która przy zastosowaniu konwencjonalnych metod fingerprintingu może osiągnąć medianę dokładności lokalizacji poniżej 2 m” \cite{Faragher2015}.

Dzięki temu BLE umożliwia implementację systemów pozycjonowania opartych na algorytmach takich jak trilateracja czy metoda centroidów. W literaturze BLE jest często podkreślane jako rozwiązanie charakteryzujące się dużym potencjałem, szczególnie w środowiskach takich jak sklepy, gdzie wymagane jest precyzyjne pozycjonowanie w ograniczonym obszarze.

\subsection{RSSI jako kluczowy wskaźnik w systemach IPS}
Siła sygnału RSSI jest podstawowym parametrem wykorzystywanym w systemach IPS opartych na BLE. Jego wartość jest obliczana na podstawie mocy sygnału odebranego przez urządzenie. W teorii wartość RSSI pozwala na precyzyjne określenie odległości od nadajnika, jednak w praktyce sygnały RSSI są bardzo niestabilne.

Fluktuacje wartości RSSI wynikają z wielu czynników:
\begin{itemize}
    \item Zakłócenia fizyczne: Ściany, meble czy elementy metalowe mogą tłumić lub odbijać sygnał co prowadzi do błędnych pomiarów.
    \item Interferencje radiowe: Inne urządzenia elektroniczne mogą zakłócać sygnał BLE, powodując jego zniekształcenia.
    \item Dynamiczność środowiska: W miejscach takich jak sklepy obecność klientów, ruch oraz zmiany w ułożeniu produktów dodatkowo destabilizują sygnał.
\end{itemize}

\subsection{Algorytmy pozycjonowania: przegląd i ograniczenia}

\subsubsection{Trilateracja – teoria i zastosowanie}
Trilateracja to metoda pozycjonowania oparta na obliczaniu odległości od co najmniej trzech nadajników BLE. Algorytm wykorzystuje geometryczne zależności między nadajnikami a odbiornikiem, aby wyznaczyć współrzędne użytkownika. Teoretycznie metoda ta zapewnia wysoką dokładność, jednak w praktyce niestabilność sygnałów RSSI znacząco obniża jej efektywność. Podobne wnioski przedstawili Gu, Lo i Niemegeers (2009): „Wyzwania związane z wielodrogowością, tłumieniem sygnału oraz interferencjami sprawiają, że algorytmy takie jak trilateracja tracą na dokładności w środowiskach wewnętrznych” \cite{Gu2009}.

\subsubsection{Metoda centroidów – zalety i wady}
Metoda centroidów jest uproszczoną alternatywą dla trilateracji, w której pozycja użytkownika jest wyznaczana jako środek ciężkości najbliższych nadajników. Choć łatwiejsza w implementacji, metoda ta cierpi na jeszcze większe ograniczenia związane z precyzją niż trilateracja, szczególnie w środowiskach o wysokiej zmienności sygnału.

\subsubsection{Fingerprinting – potencjał i wyzwania}
Fingerprinting to technika oparta na tworzeniu map referencyjnych sygnałów dla różnych lokalizacji w przestrzeni. Użytkownik jest lokalizowany na podstawie porównania aktualnych danych RSSI z wcześniej zebranymi "odciskami" sygnału. Metoda ta teoretycznie oferuje najwyższą dokładność, jednak jej głównymi ograniczeniami są:
\begin{itemize}
    \item Wysokie koszty procesu kalibracji, który wymaga zbierania danych w całym obszarze.
    \item Wrażliwość na zmiany środowiska, co wymusza częste aktualizacje map sygnałów.
\end{itemize}

Jak wskazują Liu i in. (2007): „Choć fingerprinting może osiągać wysoką dokładność, utrzymanie aktualnych odcisków sygnału w stale zmieniającym się środowisku jest trudne i kosztowne” \cite{Liu2007}.

\subsection{Wyzwania środowiskowe w kontekście IPS}

\subsubsection{Fluktuacje sygnałów RSSI w dynamicznym środowisku}
Fluktuacje RSSI wynikają z przeszkód fizycznych - przykładowo w sklepie elementy takie jak metalowe regały mogą prowadzić do niestabilnych pomiarów.

\subsubsection{Zakłócenia i interferencje radiowe}
Urządzenia elektroniczne, takie jak routery Wi-Fi, mogą powodować interferencje radiowe, które zakłócają sygnał BLE.

\subsubsection{Zmienność otoczenia – wpływ przeszkód i obecności ludzi}
Dynamiczne zmiany w otoczeniu, np. ruch klientów w sklepie, dodatkowo destabilizują odczyty RSSI.

\subsection{Filtr Kalmana – zastosowanie i efektywność}

\subsubsection{Wygładzanie sygnałów RSSI za pomocą filtru Kalmana}

Filtr Kalmana jest algorytmem stosowanym do estymacji zmiennych stanu w systemach dynamicznych, które są zakłócane przez szum. Jego szczególna efektywność wynika z połączenia predykcji modelu systemu oraz rzeczywistych obserwacji. Filtr ten znajduje zastosowanie w wielu dziedzinach, takich jak nawigacja, robotyka czy systemy pozycjonowania wewnętrznego. Zastosowanie tego filtru w IPS zostało wielokrotnie potwierdzone w literaturze. Ma i in. (2018) zauważają: „Zastosowanie filtru Kalmana do wygładzania sygnałów iBeacon pozwala znacząco obniżyć wariancję błędu pozycjonowania” \cite{Ma2018}.

Podstawowe równania filtru Kalmana można opisać następująco:

\begin{itemize}
    \item \textbf{Równanie predykcji stanu:}
    \begin{equation}
    \hat{x}_{k|k-1} = F \cdot \hat{x}_{k-1|k-1} + B \cdot u_k
    \end{equation}
    gdzie:
    \begin{itemize}
        \item $\hat{x}_{k|k-1}$ – estymowany stan w chwili $k$,
        \item $F$ – macierz przejścia stanu,
        \item $B$ – macierz wpływu sterowania,
        \item $u_k$ – wektor sterowania.
    \end{itemize}
    
    \item \textbf{Równanie predykcji macierzy kowariancji błędu:}
    \begin{equation}
    P_{k|k-1} = F \cdot P_{k-1|k-1} \cdot F^T + Q
    \end{equation}
    gdzie:
    \begin{itemize}
        \item $P_{k|k-1}$ – macierz kowariancji błędu predykcji,
        \item $Q$ – macierz kowariancji szumu procesu.
    \end{itemize}
    
    \item \textbf{Równanie aktualizacji stanu:}
    \begin{equation}
    \hat{x}_{k|k} = \hat{x}_{k|k-1} + K_k \cdot (z_k - H \cdot \hat{x}_{k|k-1})
    \end{equation}
    gdzie:
    \begin{itemize}
        \item $z_k$ – wektor obserwacji,
        \item $H$ – macierz obserwacji,
        \item $K_k$ – zysk Kalmana.
    \end{itemize}
    
    \item \textbf{Równanie aktualizacji macierzy kowariancji błędu:}
    \begin{equation}
    P_{k|k} = (I - K_k \cdot H) \cdot P_{k|k-1}
    \end{equation}
    gdzie:
    \begin{itemize}
        \item $P_{k|k}$ – zaktualizowana macierz kowariancji błędu,
        \item $I$ – macierz jednostkowa.
    \end{itemize}
\end{itemize}

Dzięki iteracyjnemu procesowi predykcji i aktualizacji filtr Kalmana pozwala na wygładzanie sygnałów RSSI, redukując fluktuacje wynikające z zakłóceń środowiskowych i szumów.

\subsubsection{Porównanie z alternatywnymi technikami filtracji}

Alternatywne metody filtracji, takie jak średnie kroczące czy filtry medianowe, są prostsze w implementacji, jednak nie oferują tak wysokiej precyzji w dynamicznych środowiskach. Na przykład średnie kroczące są skuteczne w redukcji krótkoterminowych szumów, ale nie potrafią odpowiednio reagować na dynamiczne zmiany stanu systemu.

Filtr Kalmana przewyższa te metody dzięki zdolności do adaptacji parametrów i uwzględnianiu zarówno niepewności modelu, jak i pomiarów. 
Jak zauważają Abhishek i in. (2020): „Zastosowanie filtru Kalmana w systemach pozycjonowania wewnętrznego pozwala na zmniejszenie błędów pozycjonowania o około 40\% w porównaniu do niefiltrowanych sygnałów” \cite{Abhishek2020}.

\subsubsection{Znaczenie parametrów filtru Kalmana}

Efektywność filtru Kalmana zależy od odpowiedniego dostrojenia parametrów:
\begin{itemize}
    \item \textbf{Macierz $Q$} – macierz kowariancji szumu procesu, określająca niepewność modelu.
    \item \textbf{Macierz $R$} – macierz kowariancji szumu obserwacji, opisująca niepewność pomiarów.
    \item \textbf{Macierz $F$} – macierz przejścia stanu, opisująca zależność między kolejnymi stanami.
    \item \textbf{Macierz $H$} – macierz obserwacji, przekształcająca wektor stanu na przestrzeń obserwacji.
\end{itemize}

Nieprawidłowe dostrojenie tych parametrów może prowadzić do niestabilności filtru lub błędnych estymacji. Z tego powodu konieczne jest przeprowadzenie testów i kalibracji w rzeczywistych warunkach operacyjnych. Jak wskazują Yassin i in. (2017): „Dokładne dopasowanie parametrów modelu filtru oraz uwzględnienie dynamiki środowiska jest kluczowe dla osiągnięcia optymalnych rezultatów” \cite{Yassin2017}.

\subsection{Podejście sekcyjne – uproszczenie systemu IPS}

\subsubsection{Definicja sekcji w kontekście sklepu}
Sekcja sklepu to wydzielony obszar odpowiadający fragmentowi alejki, w której produkty są przypisane w bazie danych.

\subsubsection{Optymalizacja liczby nadajników BLE na sekcję}
Dla uzyskania odpowiedniego pokrycia sygnałem optymalna liczba nadajników BLE na sekcję wynosi dwa lub trzy.

\subsubsection{Zalety sekcyjnego podejścia w środowisku sklepowym}
Podejście sekcyjne minimalizuje wymagania obliczeniowe i redukuje liczbę nadajników, jednocześnie zapewniając wysoką niezawodność systemu.

\subsection{Wnioski}
Analiza wskazuje, że podejście sekcyjne, wspierane przez filtr Kalmana, oferuje kompromis między precyzją a wydajnością w środowiskach takich jak sklepy. Redukcja wymagań dotyczących dokładności lokalizacji pozwala na znaczące obniżenie kosztów wdrożenia i utrzymania systemu IPS.


\section{Przetwarzanie języka naturalnego}

„Przetwarzanie języka naturalnego (ang. \textit{Natural Language Processing}) to dziedzina badań i zastosowań, która eksploruje, jak komputery mogą rozumieć i manipulować naturalnym językiem w formie tekstu lub mowy w celu wykonania użytecznych zadań” \\ \cite{Chowdhary2020} 


NLP znajduje zastosowanie w różnych obszarach, takich jak tłumaczenie maszynowe, przetwarzanie tekstu, streszczanie, interfejsy użytkownika, rozpoznawanie mowy i systemy ekspertowe. W szczególności w aplikacjach handlowych NLP może poprawić wyszukiwanie informacji i interakcje z użytkownikami.

Budowanie systemów NLP obejmuje analizę na kilku poziomach:
\begin{enumerate}
    \item Foniczny i fonologiczny: wymowa i dźwięk.
    \item Morfologiczny: analiza najmniejszych jednostek językowych.
    \item Syntaktyczny: struktura zdań.
    \item Semantyczny: znaczenie słów i zdań.
    \item Dyskursywny i pragmatyczny: kontekst i wiedza zewnętrzna (Liddy, 1998; Feldman, 1999). \cite{Chowdhary2020}
\end{enumerate}

Natural Language Interfaces (NLI) umożliwiają użytkownikom zadawanie pytań w języku naturalnym, co może być szczególnie przydatne w aplikacjach zakupowych, np. „Gdzie znajdę makaron?” lub „Dodaj do koszyka mleko”. W przypadku aplikacji będącej tematem pracy, NLI zostało wykorzystane również do pomocy w obsłudze aplikacji.

\subsection{Przewaga modeli językowych nad modelami generatywnymi}
Natural Language Processing (NLP), czyli przetwarzanie języka naturalnego, odgrywa kluczową rolę w budowie nowoczesnych aplikacji zorientowanych na interakcję z użytkownikiem. W porównaniu do bardziej ogólnych modeli generatywnych, takich jak ChatGPT, rozwiązania NLP oferują szereg istotnych przewag, które czynią je bardziej odpowiednimi do zastosowań w określonych domenach.

\subsubsection{Specjalizacja w przetwarzaniu języka}
NLP zostało zaprojektowane z myślą o analizie, interpretacji i przetwarzaniu języka naturalnego w sposób ukierunkowany. Pozwala to na efektywne rozpoznawanie intencji użytkownika, ekstrakcję kluczowych informacji z tekstu oraz mapowanie tych informacji na konkretne działania w aplikacji. Na przykład w aplikacjach zakupowych system NLP może łatwo rozpoznać zapytanie użytkownika takie jak „Dodaj mleko do listy zakupów” i przypisać je do odpowiedniej akcji. Modele generatywne, choć potężne, mogą być mniej precyzyjne w scenariuszach wymagających określonej interpretacji danych.

\subsubsection{Wydajność i efektywność zasobów}
Rozwiązania NLP takie jak Wit.ai, są znacznie bardziej zoptymalizowane pod względem zużycia zasobów obliczeniowych w porównaniu do dużych modeli generatywnych, takich jak ChatGPT. W przypadku aplikacji mobilnych lub działających na urządzeniach z ograniczonymi zasobami, NLP pozwala na:
\begin{itemize}
    \item szybsze przetwarzanie zapytań użytkownika,
    \item mniejsze zapotrzebowanie na moc obliczeniową,
    \item lepszą integrację z lokalnym systemem operacyjnym.
\end{itemize}

Dzięki temu rozwiązania NLP są bardziej dostępne dla szerokiego grona aplikacji użytkowych, w tym takich, które muszą działać w czasie rzeczywistym.

\subsubsection{Integracja z procesami biznesowymi}

Jednym z kluczowych atutów NLP jest jego zdolność do ścisłej integracji z procesami biznesowymi. Technologie te umożliwiają definiowanie przepływów (flows), które w sposób deterministyczny realizują zadania związane z rozpoznaną intencją użytkownika. W praktyce oznacza to, że intencje wykryte przez NLP mogą być bezpośrednio mapowane na działania, takie jak:

\begin{itemize}
    \item dodanie produktu do koszyka,
    \item generowanie przypomnień,
    \item integracja z innymi systemami.
\end{itemize}
W porównaniu, modele generatywne wymagają dodatkowego przetwarzania danych, aby zinterpretować i przekształcić wygenerowaną odpowiedź w działanie aplikacji.

\subsubsection{Kontrola i przewidywalność odpowiedzi}

Rozwiązania NLP charakteryzują się większą przewidywalnością działania. W systemach takich jak Wit.ai intencje i akcje są definiowane w sposób jawny, co pozwala na pełną kontrolę nad procesem odpowiedzi. Modele generatywne, takie jak ChatGPT, generują odpowiedzi na podstawie wzorców w danych, co sprawia, że ich działanie może być mniej przewidywalne. W aplikacjach użytkowych, gdzie kluczowe jest zaufanie użytkownika i spójność działania, deterministyczne podejście NLP jest bardziej pożądane.

\section{Obecne rozwiązania}
\label{subsec:obecne_rozwiazania}

W ramach analizy biznesowej przebadano aplikacje posiadające następujące funkcje:

\begin{itemize}
    \item Interfejs służący do nawigacji po sklepie
    \item Baza danych ze sklepami, wraz z ich lokalizacją i rozkładem produktów
    \item Asystent AI pomagający w obsłudze aplikacji
    \item Interfejs głosowy pozwalający na obsługę aplikacji przez osobę niedowidzącą
    \item System zgrywania koszyka do kodu QR w celu szybszego zakończenia zakupów
\end{itemize}
Analiza miała na celu rozpoznanie dostępnych rozwiązań na rynku, które mogłyby być inspiracją dla aplikacji będącej tematem pracy.


W tabeli \ref{tab:comparison} przedstawiono znalezione rozwiązania i ich pokrycie funkcjonalne.

\begin{table}[ht]
\centering
\scriptsize
\begin{tabular}{|l|c|c|c|c|c|}
\hline
\textbf{Aplikacja}         & \textbf{Nawigacja po sklepie} & \textbf{Baza danych sklepów} & \textbf{Asystent AI} & \textbf{Interfejs głosowy} & \textbf{Koszyk QR} \\ \hline
Walmart App                & \cmark                        & \cmark                      &                      &                           &                   \\ \hline
Target App                 & \cmark                        & \cmark                      &                      &                           &                   \\ \hline
Amazon Alexa               &                               &                             & \cmark               & \cmark                    &                   \\ \hline
Carrefour App              &                               &                             &                      &                           & \cmark            \\ \hline
IKEA Place                 & \cmark                        &                             &                      &                           &                   \\ \hline
Instacart                  &                               & \cmark                      &                      &                           &                   \\ \hline
Google Shopping            &                               &                             & \cmark               &                           &                   \\ \hline
\textbf{Shopper}           & \cmark                        & \cmark                      & \cmark               & \cmark                    & \cmark            \\ \hline
\end{tabular}
\caption{Porównanie funkcjonalności aplikacji zakupowych}
\label{tab:comparison}
\end{table}

\subsection{Walmart App}
Aplikacja Walmart App jest najblizszym rozwiązaniem do aplikacji Shopper. Posiada ona interfejs służący do nawigacji po sklepie, bazę danych ze sklepami, wraz z ich lokalizacją i rozkładem produktów. Niestety, nie posiada ona asystenta AI, interfejsu głosowego ani systemu zgrywania koszyka do kodu QR.

\subsection{Asystenci głosowi}

Asystenci głosowi, takie jak Amazon Alexa czy Google Shopping, są rozwiązaniem dla osób, które chcą zrobić zakupy bez użycia rąk. Niestety, nie posiadają one interfejsu służącego do nawigacji po sklepie, bazy danych ze sklepami, wraz z ich lokalizacją i rozkładem produktów ani systemu zgrywania koszyka do kodu QR.

\subsection{Wnioski}
Jak widać, żadna z aplikacji nie spełnia wszystkich wymagań postawionych przed aplikacją. Ponadto, wszystkie aplikacje poza Instacart i Google Assistant, są rozwiązaniem dla jednego sklepu. Aplikacja Shopper ma na celu nie tylko implementację wszystkich wymagań w ramach jednego projektu, ale też stworzenie aplikacji, która będzie działać w każdym sklepie.
