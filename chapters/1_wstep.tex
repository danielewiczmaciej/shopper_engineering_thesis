\chapter{WPROWADZENIE}
\label{chap:introduction}

Na przestrzeni ostatnich kilku dekad miał miejsce gwałtowny rozwój technologii. Przyczyniło się to do zwiększenia tempa życia każdego. Ludzie starają się optymalizować codzienne czynności, w celu odzyskania swojego czasu wolnego. W odpowiedzi na ten trend, powstaje wiele rozwiązań mających na celu usprawnienie życia codziennego ich użytkownika. 

\section{Motywacja}

\subsection{Wykluczenie społeczne}

W obecnych czasach, internet jest dostępny w każdym miejscu na Ziemi. Przyczyniło się to do zwiększenia świadomości społecznej na temat inkluzywności. Produkty wypuszczane obecnie na rynek, starają się być dostępne dla każdego. Niesety to samo nie dotyczy rozwiązań i produktów dostępnych teraz na rynku. Jednym z sektorów, gdzie nie widać postępu w dostępności dla osób niepełnosprawnych jest sektor sprzedaży detalicznej. Osoby z wadami wzroku, słuchu lub ruchu nie mogą liczyć na wiele udogodnień w trakcie robienia zakupów.

\subsection{Dynamiczny rozwój rynku aplikacji mobilnych}

Smartfony (ang. \textit{Smartphone}) są dziś w kieszeni każdego. W związku z tym, można zauważyć dynamiczny rozwój rynku aplikacji mobilnych. Firmy i deweloperzy starają się odpowiedzieć na coraz bardziej wygórowane potrzeby konsumentów.

\subsection{Brak gotowych rozwiązań}

\section{Cel pracy}

% co chcemy zrobić, implementacja papieru
% Jakie konkretne cele ma realizować praca? (np. stworzenie aplikacji ułatwiającej zakupy)
% Jakie konkretne funkcjonalności chcemy zaimplementować, inspirowane istniejącymi rozwiązaniami lub artykułami naukowymi?
% W jakich obszarach nasza aplikacja może być wykorzystana? (np. edukacja, rozrywka, badania nad interakcjami społecznymi)

% Podsumowanie rozdziałów pracy

% Krótki przegląd, co będzie zawarte w każdym rozdziale, w tym podział obowiązków autorów (jeśli dotyczy).
% Opis struktur logicznych pracy, w tym główne kroki realizacji.

\subsection{Inspiracja}

