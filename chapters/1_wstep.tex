\chapter{WSTĘP I CEL PRACY}
\label{chap:introduction}
% Wstęp - tutaj należy pokrótce opisać o co chodzi w pracy i wyraźnie wskazać cel pracy!

% Wstęp i cel pracy nakreśla problematykę opisaną lub rozwiązywaną w pracy dyplomowej wraz z uzasadnieniem celowości jej realizacji. Podaje cel i ewentualnie tezę (hipotezę). Syntetycznie opisuje dotychczasowe dokonania w danej tematyce, założenia techniczne oraz może zwięźle przedstawić zawartość poszczególnych rozdziałów. W przypadku pracy realizowanej przez kilku studentów, przy omawianiu zawartości rozdziałów należy podać ich autorów. Punkty stanowiące element składowy podrozdziału powinny być opracowane przez jednego autora

\section{Motywacja}


\section{Cel pracy}

% co chcemy zrobić, implementacja papieru
% Jakie konkretne cele ma realizować praca? (np. stworzenie aplikacji ułatwiającej zakupy)
% Jakie konkretne funkcjonalności chcemy zaimplementować, inspirowane istniejącymi rozwiązaniami lub artykułami naukowymi?
% W jakich obszarach nasza aplikacja może być wykorzystana? (np. edukacja, rozrywka, badania nad interakcjami społecznymi)

% Podsumowanie rozdziałów pracy

% Krótki przegląd, co będzie zawarte w każdym rozdziale, w tym podział obowiązków autorów (jeśli dotyczy).
% Opis struktur logicznych pracy, w tym główne kroki realizacji.

\subsection{Inspiracja}

