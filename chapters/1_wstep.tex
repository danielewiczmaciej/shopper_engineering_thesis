\chapter{WPROWADZENIE}
\label{chapter:wprowadzenie}

Na przestrzeni ostatnich kilku dekad miał miejsce gwałtowny rozwój technologii informacyjnych, który znacząco wpłynął na różne aspekty codziennego życia. W szczególności, technologie te przyczyniły się do powstania szeregu narzędzi, mających na celu wsparcie różnych grup społecznych, w tym osób niepełnosprawnych. W ramach tych rozwiązań można wyróżnić aplikacje, które ułatwiają:
\begin{itemize}
    \item \textbf{komunikację}, umożliwiając osobom niewidomym lub niedowidzącym korzystanie z usług dzięki interfejsom głosowym;
    \item \textbf{nawigację}, pozwalając na sprawne przemieszczanie się po skomplikowanych przestrzeniach, takich jak sklepy;
    \item \textbf{integrację z nowoczesnymi technologiami}, wspierając użytkowników w codziennych aktywnościach, takich jak zakupy.
\end{itemize}

W ramach niniejszego projektu dyplomowego stworzono system, który łączy te funkcjonalności, aby ułatwić zakupy osobom niewidomym oraz optymalizować doświadczenie zakupowe również dla osób zdrowych. System ten oferuje m.in.:
\begin{itemize}
    \item \textbf{nawigację po sklepie} z wykorzystaniem technologii Indoor Positioning System (IPS),
    \item \textbf{interfejs głosowy}, umożliwiający obsługę aplikacji bez potrzeby patrzenia na ekran,
    \item \textbf{generowanie kodów QR} w celu szybkiej finalizacji zakupów przy kasach samoobsługowych.
\end{itemize}

System nie tylko wspiera osoby z ograniczeniami, ale również podnosi efektywność zakupów, odpowiadając na współczesne potrzeby użytkowników i rozwój technologii mobilnych.
\section{Motywacja}

\subsection{Wykluczenie społeczne}

W obecnych czasach, internet jest dostępny w każdym miejscu na Ziemi. Przyczyniło się to do zwiększenia świadomości społecznej na temat inkluzywności. Produkty wypuszczane obecnie na rynek, starają się być dostępne dla każdego. Niesety to samo nie dotyczy rozwiązań i produktów dostępnych teraz na rynku. Jednym z sektorów, gdzie niestety nie widać postępu w dostępności dla osób niepełnosprawnych jest sektor sprzedaży detalicznej. Osoby z wadami wzroku, słuchu lub ruchu nie mogą liczyć na wiele udogodnień w trakcie robienia zakupów.
W związku z powyższym, główną motywacją stojącą za zrealizowaniem tego projektu jest chęć stworzenia aplikacji, która usprawni robienie zakupów osobom niepełnosprawnym. Jej założeniem jest ułatwienie robienia zakupów do tego stopnia, że osoba mająca problemy z poruszaniem się, widzeniem lub kontaktem z innymi ludźmi, mogłaby zrobić zakupy bez pomocy innej osoby. Znacznie wpłynęłoby to nie tylko na komfort użytkownika, ale też odbiłoby się pozytywnie na wizerunku marki sklepu, który posiada taką aplikację.

\subsection{Dynamiczny rozwój rynku aplikacji mobilnych}

Współczesne smartfony, stały się integralną częścią codziennego życia większości społeczeństwa. Ich powszechna obecność i rosnące możliwości techniczne przyczyniły się do dynamicznego rozwoju rynku aplikacji mobilnych. Zarówno firmy, jak i niezależni deweloperzy intensywnie poszukują sposobów na sprostanie coraz bardziej zaawansowanym oczekiwaniom konsumentów, wprowadzając innowacyjne rozwiązania i udoskonalając istniejące technologie.

Ten wzrostowy trend jest napędzany przez rosnące zapotrzebowanie na aplikacje ułatwiające różnorodne aspekty życia codziennego, od zarządzania czasem, przez zdrowie i fitness, po zakupy oraz dostęp do informacji. Rozwój rynku aplikacji mobilnych stwarza zatem nowe możliwości w projektowaniu narzędzi, które odpowiadają na konkretne potrzeby użytkowników w coraz bardziej spersonalizowany sposób.
\subsection{Brak gotowych rozwiązań}

Przeprowadzona analiza rynku nie wykazała dostępności gotowych rozwiązań spełniających wszystkie wymagania stawiane aplikacji. W związku z tym podjęto decyzję o opracowaniu autorskiego rozwiązania, które w pełni zaspokoi te potrzeby. Szczegółowe omówienie dostępnych rozwiązań konkurencyjnych znajduje się w sekcji \ref{subsec:obecne_rozwiazania}. Z przeprowadzonej analizy wynika, że obecne rozwiązania, takie jak aplikacje Walmart App, Amazon Alexa czy Google Shopping, oferują jedynie wybrane funkcjonalności wymagane od projektowanej aplikacji. Żadne z badanych narzędzi nie integruje w jednym systemie wszystkich kluczowych funkcji, takich jak nawigacja po sklepie, obsługa głosowa czy generowanie kodu QR na podstawie koszyka zakupów.

\section{Cel pracy}

% co chcemy zrobić, implementacja papieru
% Jakie konkretne cele ma realizować praca? (np. stworzenie aplikacji ułatwiającej zakupy)
% Jakie konkretne funkcjonalności chcemy zaimplementować, inspirowane istniejącymi rozwiązaniami lub artykułami naukowymi?
% W jakich obszarach nasza aplikacja może być wykorzystana? (np. edukacja, rozrywka, badania nad interakcjami społecznymi)
Celem pracy jest wytworzenie kompletnej aplikacji mającej na celu ułatwienie robienia zakupów. Wymagania funkcjonalne świadczące o kompletności aplikacji to:
\begin{enumerate}
    \item Interfejs służący do nawigacji po sklepie
    \item Baza danych ze sklepami, wraz z ich lokalizacją i rozkładem produktów
    \item Asystent AI pomagający w obsłudze aplikacji
    \item Interfejs głosowy pozwalający na obsługę aplikacji przez osobę niedowidzącą
    \item System zgrywania koszyka do kodu QR w celu szybszego zakończenia zakupów
\end{enumerate}
Aplikacja spełniająca powyższe wymagania ma za zadanie nie tylko usprawnić robienie zakupów przeciętnemu użytkowniku, ale przede wszystkim ułatwić tę czynność osobom niedowidzącym i seniorom. Następnymi krokami, będą nawiązanie współpracy z klientem i komercjalizacja aplikacji. Projekt ma za zadanie rozszerzyć kompetencje autorów i pozwolić na napisanie aplikacji mającej realne szanse wejścia na rynek. 



\subsection{Inspiracja}

\subsubsection{Sklepy samoobsługowe}
Na rynku istnieją już sklepy, które nie wymagają obsługi przez kasjera. Klient samodzielnie skanuje produkty i płaci za nie przy wyjściu. Przykładem takiego sklepu jest Amazon Go, lub Żabka Nano. Oba powyższe projekty działają na podobnej zasadzie.
Przed wejściem do sklepu klient musi zeskanować kod QR ze swojej aplikacji mobilnej (z przypisaną kartą płatniczą), lub przyłożyć kartę do czytnika. Po zakończeniu autoryzacji drzwi się otwierają (Żabka Nano) lub otwiera się bramka (Amazon Go). W całym sklepie na suficie umieszczone są kamery.
Ich zadaniem jest monitorowanie każdego z produktów. Kiedy klient zdejmuje produkt z półki, kamera rejestruje ten fakt. Kiedy klient odłoży produkt z powrotem na półkę, kamera rejestruje również to rejestruje. Po zakończeniu zakupów, klient wychodzi ze sklepu. System automatycznie oblicza wartość zakupów i ściąga odpowiednią kwotę z konta lub karty klienta.

Taki model sklepu jest inspiracją dla jednego z elementów aplikacji wytworzonego w ramach tej pracy. Jest to system skanowania koszyka do kodu QR, który pozwala na szybsze zakończenie zakupów. 

\section{Struktura pracy i zakres tematyczny}

W dalszej części pracy szczegółowo omówiono, w jaki sposób zidentyfikowane wymagania zostały zaadresowane w ramach projektowanej aplikacji. Rozdział 2 zawiera analizę problemu oraz istniejących rozwiązań rynkowych, które stanowiły punkt wyjścia dla określenia założeń projektowych. Kolejne rozdziały skupiają się na opisie zaproponowanego rozwiązania, jego architektury, implementacji poszczególnych funkcjonalności oraz przeprowadzonych testów, mających na celu weryfikację skuteczności zaprojektowanego systemu.
